\section{Ablaufumgebung}\label{sec:ablaufumgebung}

\subsection{Voraussetzungen}\label{subsec:voraussetzungen}

\subsubsection{Maven}

Um das Projekt lokal aufzusetzen, wird \textit{Maven} in der Version \texttt{3.6} oder h\"oher ben\"otigt.
Ob bzw. in welcher Version \textit{Maven} bereits installiert ist, kann mithilfe des folgenden Befehls ermittelt werden:

\begin{lstlisting}[label={lst:mvnv}]
mvn -v
\end{lstlisting}




\subsubsection{MySQL}

Zus\"atzlich ist für die Datenverwaltung eine lokale \textit{MySQL}-Datenbank auf Port \texttt{3306} (default) erforderlich,
in der f\"ur den Nutzer \texttt{ROOT} das folgendes Passwort festzulegen ist: \texttt{Test\#-\#44}

Um die Datenbank aufzusetzen, gibt es wiederum zwei M\"oglichkeiten, die sich im Rahmen der Entwicklung als gleichermaßen geeignet erwiesen:

\begin{itemize}
    \item \textbf{Manuell:} Einrichtung mithilfe des MySQL Installers (\url{https://dev.mysql.com/downloads/}).
    \item \textbf{Docker:} Starten von MySQL als Docker Container.
    Folgender Befehl kann hierf\"ur genutzt werden:
\end{itemize}

\begin{lstlisting}[label={lst:dockerdb}]
docker run -p 3306:3306 -e MYSQL_ROOT_PASSWORD="Test#-#44" --name vdb -d mysql
\end{lstlisting}

Die Parameter \texttt{--name <...>} zur Namensgebung sowie \texttt{-d} zum Starten im detached-Modus sind empfohlen, aber optional.


\subsubsection{Tomcat Server}

Voraussetzung für die REST-API bzw. die Webanwendung ist zudem \textit{Tomcat Server} (getestet in Version \texttt{9.0.40}).

\subsection{Bauen der Anwendungen}\label{subsec:bauen-der-anwendungen}

Zum Intallieren aller Abhängigkeiten und zum Bauen der Module ist folgender Befehl auszuf\"uhren:

\begin{lstlisting}[label={lst:mvncleaninstall}]
mvn clean install
\end{lstlisting}

Mithilfe des Plugins \textit{frontend-maven-plugin}, genutzt im REST-Konfigurationsmodul, 
wird darüber hinaus in diesem Kontext direkt die Webanwendung gebaut.
Hierfür werden lokal \textit{Node/NPM} sowie \textit{Angular} installiert und
im Anschluss alle Abhängigkeiten der Frontendanwendung geladen und das Projekt gebaut. Ablage unter: 
\newline\texttt{./configuration.rest/src/main/webapp/app}

Somit sind \textit{Angular} bzw. \textit{Node/NPM} nur für die Entwicklung erforderlich, nicht aber zum Bauen aller Module. 
Alle für diese lokale Konfiguration geladenen Dateien werden in \texttt{./configuration.rest/node} gespeichert
(dieser Ordner wird von der Versionsverwaltung ignoriert).



\subsection{Starten der Anwendung}

\subsubsection{Konsolenoberfläche (veraltet)}

Um die als Konsolenanwendung implementierte Konfiguration zu starten, ist die \texttt{main}-Methode der 
Klasse \texttt{ConfigurationSpringImpl.java} des Moduls \texttt{configuration} als Java-Anwendung zu starten.

\subsubsection{REST-API und Webanwendung}

Über das Modul \texttt{configuration.rest} lässt sich ein Server starten, der die REST-Schnittstelle sowie die Weboberfläche bereitstellt.
Hierfür ist die für dieses Modul generierte \texttt{war}-Datei in einer Tomcat-Konfiguration als zu deployendes Artefakt anzugeben.
Für die Nutzung in \textit{IntelliJ} liegt im Repository unter \texttt{./.run} eine direkt nutzbare Konfiguration für Tomcat \texttt{9.0.40} bereit.

Die von dem Modul bereitgestellte Webkonfiguration
\linebreak(\texttt{./configuration.rest/src/main/webapp/WEB-INF/web.xml}) sieht folgende Pfade vor:

\begin{itemize}
    \item \textbf{\texttt{<host>:<port>/api/...}}: REST-Schnittstelle
    \item \textbf{\texttt{<host>:<port>/app/...}}: Webanwendung
\end{itemize}

Ein großer Vorteil des Deployments auf demselben Server liegt darin, auf diesem Weg Cross-Origin-Zugriffe auszuschließen.
Für die lokale Entwicklung kann allerdings im Konstruktor der Klasse \texttt{ConfigurationRestEasyImpl} ein CORS-Filter konfiguriert werden.
Der Code hierfür ist im produktiven System jedoch aus Sicherheitsgründen auszukommentieren.

\begin{lstlisting}[label={lst:cors}]
CorsFilter corsFilter = new CorsFilter();
corsFilter.getAllowedOrigins().add("*"); // for dev mode only!
corsFilter.setAllowedMethods("OPTIONS, GET, POST, DELETE, PUT, PATCH");
SINGLETONS.add(corsFilter);
\end{lstlisting}